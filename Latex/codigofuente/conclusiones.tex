\chapter{Conclusiones y consideraciones}

\section{Conclusiones}

En el desarrollo del presente trabajo de grado se encontraron algunos aspectos que describen la contaminación del aire de Bogotá, en los años 2018 a 2020, específicamente de los contaminantes $PM_{10}, PM_{2.5} $ y $O_3$. En primer lugar se evidenció, que durante estos tres años, se sobrepasó la norma establecida por la secretaria de ambiente de Bogotá en mas del $20\%$ de los días analizados, lo cual es una alerta que se debe considerar en la toma de desiciones para los planes de descontaminación del aire de Bogotá, además, en las series de tiempo presentadas en la gráfica \ref{seriesdet}, se evidencia el crecimiento de los niveles de contaminación hacia el final del año 2020, año en el cual se mostró una mejoria ocasionada por las restricciones establecidas por la pandemia causada por el COVID 19. 

Se evidencia en el capítulo 2, que la modelación del comportamiento de los datos de contaminación en el aire, mejora considerablemente al tener en cuenta algún punto de cambio y es algo interesante observar la manera en que estos puntos de cambio coinciden justo con alertas decretadas por la secretaria de ambiente de Bogotá, o con días en los cuales hubo día sin carro en la ciudad.

El objetivo principal del presente trabajo de grado era determinar la correlación entre dos contaminantes relacionados con el aire de la ciudad de Bogotá, luego de construir los modelos bivariados para las tres posibles parejas de contaminantes y de encontrar su correlación a partir del parámetro $\theta$ se obtuvieron algunos resultados que vale la pena resaltar, como por ejemplo que los contaminantes $PM_{10}$ y $O_3$ tienen una correlación negativa y muy cercana a $-1$, lo cual indica que en los lugares de Bogotá, en los que hay altos niveles de concetración de material particulado de $10$ micrometros de tamaño, la cantidad de particulas de Ozono, se reducen o al contrario, si se observa una cantidad elevada de Ozono, el material partículado aumenta. Esto lo que puede sugerir es que las estrategias para contrarestar los altos niveles de concentración, pueden estar fallando, en el sentido que, si bien es cierto que se disminuye un contaminante, se está fortaleciendo el otro, lo cual no es lo esperado. De la misma manera, para el par $PM_{2.5}$ vs $O_{3}$ sucede igual, la concentración de una reduce la del otro. 

Para el par de contaminantes material particulado de $2.5 \mu m$ y $10 \mu m$ de tamaño, sucede matemáticamente, justo lo esperado y es que por ser el mismo material, pero con distinto tamaño, la lectura de su concentración en las estaciones de monitoreo deberían estar correlacionadas pues uno esta "contenido" dentro del otro. En efecto la correlación presentada por el modelo realizado fue de $\theta=0.7546785$ tal como se observa en la tabla \ref{infoestad_PM10_PM2.5}. 


\section{Consideraciones}

El modelo bivariado realizado, arrojo valores de correlación para el ozono vs cada uno de los dos materiales partículados muy exactos, esta correlación puede llegar a ser mas exacta si se tienen en cuenta los otros dos modelos realizados de manera univariada, es decir, ajustar el modelo y buscar la correlación de los dos procesos teniendo en cuenta un o dos puntos de cambio y porqué no, más momentos en los cuales hubo cambios extremos. 