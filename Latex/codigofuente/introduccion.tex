\chapter*{Introducción}


La asociación o correspondencia entre dos variables, ha sido estudiada desde distintos puntos de vista y probada por diferentes autores. De acuerdo con Farlie (1960) en \cite{farlie} algunos de estos coeficientes de correlación son: 

\begin{enumerate}
\item Coeficiente de correlación de Pearson.
\item Coeficiente de correlación de Spearman.
\item Coeficiente de correlación de Kendall $\tau$
\item Probabilidad en concordancia. 
\end{enumerate}

Farlie (1960), afirma en \cite{farlie} que la eficiencia de cada uno de estos coeficientes no es muy conocida, sin embargo, es que ninguno es uniformemente mejor que el otro. De manera particular, el coeficiente de correlación es una herramienta útil. Sólo en algunos casos, se utiliza la Cópula de Farlie-Gumbel-Morgenstern (FGM) para determinar o estimar el parámetro $\theta$ que representará la correlación entre dos parejas de contaminantes, estos son $PM_{10}$ vs $O_3$, $PM_{2.5}$ vs $O_3$ y $PM_{2.5}$ vs $PM_{10}$, para poder hacer esto, se hará de la siguiente manera. En el capítulo 1 se describirán algunos elementos de la inferencia bayesiana como el teorema de bayes, la creación de cadenas de Markov mediante el método de Monte Carlo (MCMC) y los algoritmos utilizados para lograr generar dichas cadenas. 

También se presentan algunos criterios de convergencia para los parámetros obtenidos mediante las simulaciones por el método MCMC, esto para determinar su validez y generar confianza en el módelo, estos criterios de convergencia son el de Geweke, que analiza dos partes de las cadenas construidas, el $10\%$ inicial y el $50\%$ final, verificando que la diferencia de sus medias sea mínima, cuando esto sucede, los parámetros son aceptados. De la misma manera se estudian otros criterios de convergencia como lo son el de Heidelberger-Welch, el de Raftery and Lewis y el de Gelman $\&$ Rubin. También se hace una descripción detallada del concepto de cópula junto con sus propiedades y conceptos previos necesarios para entenderlas y al finalizar este capítulo se describe la cópula utilizada para determinar la correlación de los tres pares de contaminantes. \\

En el segundo capítulo se abordan las características de los datos analizados. Para esto se inicia con una descripción de cada material contaminante junto con la reglamentación establecida por el ministerio de ambiente y desarrollo sostenible de Colombia. Esta descripción, está acompañada de los porcentajes en los cuales, durante el periodo comprendido entre el 2018 y el 2020, estos tres contaminantes analizados rebasan la norma establecida y se representa todo esto en tres series de tiempo diferentes construidas con el software R. Adicionalmente, se realiza el análisis de la gráfica de medias acumuladas que representa los datos de cada contaminante y se empieza a elaborar un modelo con el fin de encontrar los parámetros que se ajustan de la mejor forma a los datos conocidos. Se realizan tres modelos diferentes buscando un ajuste óptimo de los datos. El primero, \textbf{sin puntos de cambio}, no tiene en cuenta alguna afectación ocurrida durante los tres años, que muestre un cambio brusco en el comportamiento de los datos. En el segundo análisis, con \textbf{un punto de cambio}, se tiene en cuenta uno de los cambios mas representativos que hay en la serie de tiempo, para tomarlo como punto de cambio y hacer que el modelo cambie su comportamiento en ese punto dado. De igual manera en el tercer análisis \textbf{dos puntos de cambio}, se observa mejor el ajuste, dado que se configura el código programado para que asuma dos puntos específicos en los que hubo algún comportamiento drástico en los datos.\\

El capítulo tres, corresponde al diagnóstico de convergencia para verificar si los resultados obtenidos en el capítulo dos, eran óptimos o debían ser modificados, para cada uno de los tres modelos se realizaron los tests hasta obtener resultados favorables que permitieran parámetros adecuados. Estos resultados dan paso al cuarto capítulo, donde se toman como referencia los parámetros obtenidos para el primer ajuste, \textbf{sin puntos de cambio} y se comienza el análisis bivariado. Buscando, mediante la cópula utilizada, el parámetro que determinará la correlación entre cada par de contaminantes, aquí se presentan los resultados numéricos obtenidos sobre esta correlación, encontrando cosas partículares sobre cada par de contaminantes. \\

Finalmente, en el capítulo número cinco, se presentan algunas conclusiones de todo lo realizado en el desarrollo del presente trabajo de grado y se sugiere además la realización del modelo bivariado teniendo en cuenta los dos ajustes faltantes. 